%\PassOptionsToPackae{gray}{xcolor}
\documentclass[hyperref={pdfpagelabels=false},12pt]{beamer}
\setbeamertemplate{frametitle}[default][center]
\mode<presentation>
{
 \usetheme{Warsaw}      % or try Darmstadt, Madrid, Warsaw, ...
 \usecolortheme{default} % or try albatross, beaver, crane, ...
 \usefonttheme{default}  % or try serif, structurebold, ...
 \setbeamertemplate{footline}[frame number]
 \setbeamertemplate{caption}[numbered]
}

\usepackage[utf8]{inputenc}
\usepackage{mathtools}
\usepackage{bm}
\usepackage{helvet}
\usepackage{listings}
\usepackage{gensymb}
\usepackage{array}
\usepackage{times}
\usepackage{xcolor}
\usepackage{default}
\usepackage{ulem}
\usepackage{minted}
\usepackage{hyperref}
\usepackage{booktabs}

% Great Commands
\newcommand{\ig}[2]{\includegraphics[width=#1\linewidth]{#2}}
\newcommand{\mybutton}[2]{\hyperlink{#1}{\beamerbutton{{#2}}}}
\newcommand{\myvbutton}[2]{\vfill\hyperlink{#1}{\beamerbutton{{#2}}}}
\newcommand{\code}[2]{\mintinline{#1}{#2}}
\newcommand{\haskell}[1]{\mintinline{haskell}{#1}}
\newcommand{\unnamedUrl}[1]{\href{#1}{\color{blue}{#1}}}
\newcommand{\namedUrl}[2]{\href{#1}{\color{blue}{#2}}}
\newcommand{\pygment}[3]{\inputminted[bgcolor=lightgray,linenos,fontsize=#1]{#2}{#3}}
\newcommand{\pygmentLines}[5]{\inputminted[bgcolor=lightgray,linenos,fontsize=#1,firstline=#2,lastline=#3,autogobble]{#4}{#5}}

% Color Scheme
\definecolor{pittblue}{RGB}{28,41,87}
\definecolor{pittgold}{RGB}{205,184,125}
\setbeamercolor{structure}{fg=pittgold}
\setbeamercolor{button}{bg=pittblue}

\title[Parsing]{{Baby's First Parsing with Haskell and Trifecta}}
\author[Parsing]{{Barry Moore II}}
\institute[CRC]{Center for Research Computing \\ University of Pittsburgh}
\date{}

\beamertemplatenavigationsymbolsempty

\begin{document}

\begin{frame}[plain]
\titlepage
\end{frame}

\begin{frame}{Outline}
  \begin{itemize}
    \item Applicative Functors
    \item Introduction to Trifecta and Parsers
    \item In-depth Examples
    \begin{itemize}
      \item Semantic Versions
      \item IPv4 Addresses
      \item United States Telephone Numbers
    \end{itemize}
  \end{itemize}
\end{frame}

\begin{frame}{Functor}
  \pygment{\normalsize}{haskell}{code/Functor.hs}
  \begin{itemize}
    \item The function \haskell{(a -> b)} is ``lifted'' into the structure \haskell{f}
  \end{itemize}  
  \pygment{\normalsize}{haskell}{code/FunctorLift.hs}
\end{frame}

\begin{frame}{Functor Example}
  \pygment{\normalsize}{haskell}{code/FunctorExample.hs}
  \begin{itemize}
    \item \haskell{f :: []}
    \begin{itemize}
      \item List is a Functor!
    \end{itemize}
    \item \haskell{a :: Integer}
    \item \haskell{(a -> b) :: Integer -> Integer}
    \item Result: \haskell{[3, 4, 5]}
  \end{itemize}
\end{frame}

\begin{frame}{Applicative Functor}
  \pygment{\normalsize}{haskell}{code/Applicative.hs}
  \vspace{-0.5cm}
  \begin{itemize}
    \item \haskell{(<*>)} is pronounced ``apply''
    \item Here, our \haskell{(a -> b)} is embedded in structure, another list example:
  \end{itemize}
  \pygment{\normalsize}{haskell}{code/ApplicativeExample.hs}
  \vspace{-0.5cm}
  \begin{itemize}
    \item \haskell{(+) :: Integer -> Integer -> Integer}
    \item \haskell{(1+) :: Integer -> Integer}
  \end{itemize}
\end{frame}

\begin{frame}{Parser Combinators}
  \begin{itemize}
    \item Parser Combinators Bible? (Graham Hutton and Erik Meijer): \unnamedUrl{http://www.cs.nott.ac.uk/~pszgmh/monparsing.pdf}
    \item Packages and listed maintainer (design is quite similar amongst them):
    \begin{itemize}
      \item Trifecta (Edward Kmett): \unnamedUrl{https://hackage.haskell.org/package/trifecta}
      \item Attoparsec (Bryan O'Sullivan) \unnamedUrl{http://hackage.haskell.org/package/attoparsec}
      \item Megaparsec (Mark Karpov) \unnamedUrl{http://hackage.haskell.org/package/megaparsec}
      \item Parsec (Herbert Valerio Riedel) \unnamedUrl{http://hackage.haskell.org/package/parsec}
    \end{itemize}
  \end{itemize}
\end{frame}

\begin{frame}{Parser Combinators Cont.}
  \begin{itemize}
    \item Parsing is actually a two step process
    \begin{enumerate}
      \item ``parse'' the text
      \item ``marshalling'' into a useful datatype
    \end{enumerate}
  \end{itemize}
\end{frame}

\begin{frame}{Parser Combinators Example}
  \setbeamercovered{invisible}
  \onslide*<1,5>{
    \pygment{\scriptsize}{haskell}{code/rational/src/Main.hs}
  }
  \onslide*<2>{
    \pygmentLines{\scriptsize}{6}{7}{haskell}{code/rational/src/Main.hs}
    \begin{itemize}
      \item \haskell{Parser a} and \haskell{Result a} both have Monad instances
      \item \haskell{a} is the marshalled datatype
      \item \haskell{parseString} comes from Trifecta
      \item \haskell{mempty} is a \haskell{Text.Trifecta.Delta}
      \begin{itemize}
        \item A ``Delta'' keeps track of the cursor position of the parser,
          so it can be referred to later, e.g. in error messages.
      \end{itemize}
    \end{itemize}
  }
  \onslide*<3>{
    \pygmentLines{\scriptsize}{9}{13}{haskell}{code/rational/src/Main.hs}
    \begin{itemize}
      \item \texttt{1 \% 2} forms the rational $\frac{1}{2}$
      \item \haskell{decimal} parses a single number (think \haskell{Parser Integer})
      \item Use \haskell{fmap} to partially apply, \haskell{Parser (Integer -> Rational)}
      \item \haskell{(*>)} is the sequence operator for \haskell{Applicative},
          it completes both parsers and keeps the result of the second
    \end{itemize}
  }
  \onslide*<4>{
    \pygmentLines{\scriptsize}{15}{17}{haskell}{code/rational/src/Main.hs}
    \begin{itemize}
      \item The \haskell{main} function is the entry point for Haskell
      \item \haskell{print} will output a \haskell{Result Rational}
      \item \haskell{p} takes a \haskell{Parser Rational} and a \haskell{String}
      \item Result: \texttt{Success (1 \% 1)}
    \end{itemize}
  }
\end{frame}

\begin{frame}{Parser Combinators Example Cont.}
  \setbeamercovered{invisible}
  \begin{itemize}
    \item Does anyone see an issue with this parser?
    \pygmentLines{\scriptsize}{9}{13}{haskell}{code/rational/src/Main.hs}
    \uncover<2->{
      \item What if we pass it \haskell{"1/0"}?
    }
    \uncover<3->{
      \pygmentLines{\scriptsize}{9}{17}{haskell}{code/rational/src/Main.1.hs}
    }
  \end{itemize}
\end{frame}

\begin{frame}{Useful Combinators Documentation}
  \begin{itemize}
    \item Trifecta uses parsers from the \texttt{parsers} package (also Edward Kmett):
    \begin{itemize}
      \item \unnamedUrl{https://hackage.haskell.org/package/parsers-0.12.10/docs/Text-Parser-Char.html}
      \item \unnamedUrl{https://hackage.haskell.org/package/parsers-0.12.10/docs/Text-Parser-Combinators.html}
      \item \unnamedUrl{https://hackage.haskell.org/package/parsers-0.12.10/docs/Text-Parser-Token.html}
    \end{itemize}
  \end{itemize}
\end{frame}

\begin{frame}{Notes on Examples}
  \begin{itemize}
    \item Each example is a separate Stack project:
    \begin{itemize}
      \item Get \code{bash}{stack}: \unnamedUrl{https://docs.haskellstack.org/en/stable/README/}
      \item Run each project: \code{bash}{stack run}
    \end{itemize}
    \item Rational Parser: \code{bash}{code/rational}
    \item Semantic Version Parser: \code{bash}{code/semanticVersions}
    \item IPv4 Address Parser: \code{bash}{code/ipv4Address}
    \item US Phone Number Parser: \code{bash}{code/phoneNumber}
  \end{itemize}
\end{frame}

\begin{frame}{Semantic Version Exercise: Problem and Types}
  \setbeamercovered{invisible}
  \pygmentLines{\scriptsize}{58}{65}{haskell}{code/semanticVersions/src/Main.hs}
  \uncover<2->{
    \pygmentLines{\scriptsize}{22}{30}{haskell}{code/semanticVersions/src/Main.hs}
  }
\end{frame}

\begin{frame}{Semantic Version Exercise: Types Cont.}
  \pygmentLines{\normalsize}{7}{20}{haskell}{code/semanticVersions/src/Main.hs}
\end{frame}

\begin{frame}{Semantic Version Exercise: Parser}
  \setbeamercovered{invisible}
  \onslide*<1>{
    \pygmentLines{\normalsize}{32}{39}{haskell}{code/semanticVersions/src/Main.hs}
  }
  \onslide*<2>{
    \pygmentLines{\scriptsize}{48}{54}{haskell}{code/semanticVersions/src/Main.hs}
    \begin{itemize}
      \item Goal, parse either:
      \begin{itemize}
        \item a number which doesn't start with \haskell{0} 
        \item The number \haskell{0}
      \end{itemize}
      \item \haskell{fmap read ::} \haskell{Parser String -> Parser Integer}
    \end{itemize}
  }
  \onslide*<3>{
    \pygmentLines{\scriptsize}{41}{44}{haskell}{code/semanticVersions/src/Main.hs}
    \begin{itemize}
      \vspace{-0.5cm}
      \item \haskell{optional p :: Parser (Maybe a)}
      \item \haskell{optionally ::} \haskell{Parser (Maybe [NumberOrString]) ->} \haskell{Parser [NumberOrString]}
      \item \haskell{symbol} -- parses string and skips trailing whitespace
    \end{itemize}
    \pygmentLines{\scriptsize}{55}{56}{haskell}{code/semanticVersions/src/Main.hs}
  }
  \onslide*<4>{
    \pygmentLines{\scriptsize}{45}{47}{haskell}{code/semanticVersions/src/Main.hs}
    \begin{itemize}
      \item Extremely similar to \haskell{prerelease}
    \end{itemize}
  }
\end{frame}

\begin{frame}{Semantic Version Exercise}
  \Huge \code{bash}{stack run}
\end{frame}

\begin{frame}{IPv4 Address Exercise: Problem and Types}
  \setbeamercovered{invisible}
  \pygmentLines{\scriptsize}{40}{45}{haskell}{code/ipv4Address/src/Main.hs}
  \uncover<2->{
    \pygmentLines{\scriptsize}{8}{12}{haskell}{code/ipv4Address/src/Main.hs}
  }
\end{frame}

\begin{frame}{IPv4 Address Exercise: Parser}
  \setbeamercovered{invisible}
  \onslide*<1>{
    \pygmentLines{\normalsize}{17}{18}{haskell}{code/ipv4Address/src/Main.hs}
  }
  \onslide*<2>{
    \pygmentLines{\scriptsize}{20}{23}{haskell}{code/ipv4Address/src/Main.hs}
    \begin{itemize}
      \item \haskell{shift v s} -- shift \haskell{v} left by \haskell{s} bits
      \item fold shifted values onto \haskell{0} to generate a \haskell{Word32}
    \end{itemize}
  }
  \onslide*<3>{
    \pygmentLines{\scriptsize}{30}{38}{haskell}{code/ipv4Address/src/Main.hs}
    \begin{itemize}
      \item \haskell{octets} -- list of 4 values, all \haskell{<= 255}
    \end{itemize}
  }
  \onslide*<4>{
    \pygmentLines{\scriptsize}{24}{29}{haskell}{code/ipv4Address/src/Main.hs}
  }
\end{frame}

\begin{frame}{IPv4 Address Exercise}
  \Huge \code{bash}{stack run}
\end{frame}

\begin{frame}{US Phone Number Exercise: Problem and Types}
  \setbeamercovered{invisible}
  A bit more annoying in Applicative than the others...
  \pygmentLines{\scriptsize}{78}{86}{haskell}{code/phoneNumber/src/Main.hs}
  \uncover<2->{
    \pygmentLines{\scriptsize}{11}{16}{haskell}{code/phoneNumber/src/Main.hs}
  }
\end{frame}

\begin{frame}{US Phone Number Exercise: Parser}
  \setbeamercovered{invisible}
  \onslide*<1,4>{
    \pygmentLines{\scriptsize}{51}{53}{haskell}{code/phoneNumber/src/Main.hs}
  }
  \onslide*<2>{
    \pygmentLines{\scriptsize}{64}{65}{haskell}{code/phoneNumber/src/Main.hs}
    \begin{itemize}
      \item Either:
      \begin{itemize}
        \item country code ``pattern'': +1-\textbf{123}
        \item area code: \textbf{123}
        \item area code w/ parenthesis: (\textbf{123})
      \end{itemize}
    \end{itemize}
    \pygmentLines{\scriptsize}{57}{62}{haskell}{code/phoneNumber/src/Main.hs}
  }
  \onslide*<3>{
    \pygmentLines{\scriptsize}{55}{56}{haskell}{code/phoneNumber/src/Main.hs}
    \pygmentLines{\scriptsize}{63}{63}{haskell}{code/phoneNumber/src/Main.hs}
    \begin{itemize}
      \item \haskell{nDigits} -- return \haskell{Parser Integer} with \haskell{n} digits
    \end{itemize}
  }
  \onslide*<5>{
    \pygmentLines{\scriptsize}{67}{75}{haskell}{code/phoneNumber/src/Main.hs}
  }
  \onslide*<6>{
    \pygmentLines{\scriptsize}{51}{53}{haskell}{code/phoneNumber/src/Main.hs}
    \vspace{-0.5cm}
    \pygmentLines{\scriptsize}{76}{76}{haskell}{code/phoneNumber/src/Main.hs}
    \vspace{-0.5cm}
    \begin{itemize}
      \item \haskell{f :: Parser} \haskell{(Exchange -> LineNumber -> PhoneNumber)}
      \item \haskell{a :: Parser (Exchange, LineNumber)}
      \item \haskell{uncurry :: (a -> b -> c) -> (a, b) -> c}
      \item \haskell{c :: Parser PhoneNumber}
    \end{itemize}
  }
\end{frame}

\begin{frame}{US Phone Number Exercise}
  \Huge \code{bash}{stack run}
\end{frame}

\begin{frame}{Wrapping Up}
  \begin{itemize}
    \item HPFPP used the Monadic \haskell{do} syntax everywhere
    \item However, \haskell{Applicative} is enough in many cases
    \item Parser Combinators feel very powerful in Haskell, but error messages are frequently cryptic
    \item Trifecta is a nice introduction, but my understanding is other libraries can be more efficient
    \item Code: \unnamedUrl{https://github.com/barrymoo/parser-combinators-haskell-trifecta-intro}
    \begin{itemize}
        \item To do: add HUnit tests
        \item Question: is there a library that creates automatic failure cases for parsers?
    \end{itemize}
  \end{itemize}
\end{frame}

\end{document}
